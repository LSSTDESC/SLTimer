%
% ======================================================================
\RequirePackage{texmf/styles/docswitch}
% \flag is set by the user, through the makefile:
%    make note
%    make apj
% etc.
\setjournal{\flag}

\documentclass[\docopts]{\docclass}

% You could also define the document class directly
%\documentclass[]{emulateapj}

% Custom commands from LSST DESC, see texmf/styles/lsstdesc_macros.sty
\usepackage{texmf/styles/lsstdesc_macros}
\usepackage{graphics, graphicx}
\usepackage{multirow}
\graphicspath{{./}{./figures/}}
\pdfmapfile{+mathpple.map} % Seems Travis needs this, for some reason!
\bibliographystyle{apj}
\pdfmapfile{+mathpple.map}


% Add your own macros here:



%
% ======================================================================

\begin{document}

\title{ Estimating Time Delay Measurement Feasibility in TDC2 }

\maketitlepre

\begin{abstract}

We use the 10 gateway dataset light curves and the PyCS curve-shifting analysis code to estimate time delay bias and uncertainty. By noticing the extra noise introduced by sampling on highly-fluctuating quasar light curve, we further introduce a new parameter $\sigma_{intrinsic}$, which rescale the noise by $\sigma_{new}=\sqrt{\sigma_{intrinsic}^2+\sigma_{old}^2}$. We also incorporate the lens model prior in our analysis.  Finally, we find that even with this crude model, we can measure time delay to $2\%$ accuracy and $5\%$ precision. 


\end{abstract}

% Keywords are ignored in the LSST DESC Note style:
\dockeys{latex: templates, papers: awesome}

\maketitlepost

% ----------------------------------------------------------------------
%

\section{Introduction}
\label{sec:intro}
The second time delay challenge (hereafter as TDC2) is a proposed project to investigate how accurate time delay can we measure from multi-filter light curve data.  TDC2 team has made ten gateway data. The goal of this paper is to analyze these data and answer the following questions:
\begin{enumerate}

\item Whether simple lenses model could help us measure time delay better?
\item What kind of whitening method, a method to make the data as if they were taken from the same filter, should we use?
\item Is it possible to measure time delay to $0.2\%$ accuracy and $0.5\%$ precision? 

The paper will be structured as follows.  In section \ref{sec:method}, we will describe the method we use, and show the result in section \ref{sec:results}. In section \ref{sec:discussion}, we will discussed the results and then, in section \ref{sec:conclusions}, we summarize our conclusions.  

\end{enumerate}

%------------------------------------------------------------------------
\section{Method}
\label{sec:method}

In this section, we describe the process of our analysis step by step.

\subsection{Whitening method}

The main difference between TDC2 and TDC1 data is that TDC2 tries to model the light curve taken by six different filters, which add additional variability to the light curves. We adopt the whitening method, which offsets the light curve to get a set of points that look more like they were taken in one filter. We test two kinds of whitening method. The first one (Whitening method 1 hearafter) is to offset  the magnitude of whole light curves to a common mean. The second one (Whitening method 2 hearafter) does the same thing as the first one, but instead of offsetting all the light curves, it offset light curves season by season. Obviously, the second one looks more reasonable because we don't expect that lightcurves in different bands have long term relations with each others. However, the second whitening method will also introduce additional fluctuations to light curves, and will probably wash out the signal.


\subsection{Additional Noise Model}
We notice that the error in the lightcurve can be larger to the reported error in TDC2 data. For example, the data in different filters will have different mean amplitude. Though the difference is partially corrected by the whitening method, it will make our light curve noisier. The micro-lensing variability could also be different for data in different filters. Besides, the light curves are sampled at an average time step 4 days, and quasars might have some high frequency fluctuation, which will also add additional noise to our light curve.

To address this issue, we assume a simple model with only one parameter $\sigma_{intrinsic}$. The model rescales the noise in magnitude by the following formula, $\sigma_{new}=\sqrt{\sigma_{intrinsic}^2+\sigma_{old}^2}$.  Physically, it models the additional noise by a scale-free, uncorrelated log normal scatter light curve.

\subsection{PyCS}
PyCS is a package that implements free knot spline technique described by \cite{2013A&A...553A.120T}. It models both intrinsic quasar fluctuation and also fluctuation caused by micro-lensing with two spline functions. With the use of the PyCS package, we can compute the $\chi^2$ of the fitting for a given time delay $\Delta t$.

To analyze the data, we compute likelihood on 1000 time delay grids from -150 days to 150 days.  For each time delay, we can compute a $\chi^2$. The log likelihood function, defined as $log(P(data \mid \Delta t))$, could be computed by the following formula,
\begin{equation}
log(P(data \mid \Delta t))  = log (\Sigma_i  \frac{1}{\sqrt{2 \pi \sigma_i^2}} ) -\frac{1}{2} \chi^2,
\end{equation}
where i runs all over the data points.

\subsection{Lens Prior}
The TDC2 data will give us an approximated fermat potential $\Delta \phi$ with some error, which can be used to compute time delay prior $P(\Delta t)$.

First, we know $P(\Delta t)$ can be calculated by the following equation.
\begin{align}
P(\Delta t) &= \int \int P(\Delta t, H_0, \Delta \phi) dH_0 d \Delta \phi \\
&= P(\Delta t | H_0, \Delta \phi) P(H_0)P(\Delta \phi) dH_0 d\Delta \phi
\end{align}

Time delay $\Delta t$ is related to the fermat potential $\Delta \phi$ by,

\begin{equation}
\Delta t = \frac{D_{\Delta t}}{c} \Delta \phi
\end{equation}

And we know $D_{\Delta t}$ is inverse proportional to $H_0$, so we can relate $D_{\Delta t}$ to $H_0$ by the following formula,
\begin{equation}
D_{\Delta t}= \frac{Q}{H_0}
\end{equation}
Q will be given in TDC2 data, and by change of scale we can make $c =1$

From the equation $\Delta t = \frac{Q}{H_0}\Delta \phi$,  we know
$P(\Delta t | H_0, \Delta \phi) = \delta(\Delta t - \frac{Q}{H_0}\Delta \phi)$.
Then we get

\begin{align}
P(\Delta t) = \int\int \delta(\Delta t - \frac{Q \Delta \phi }{H_0}) P(H_0)P(\Delta \phi) dH_0 d\Delta \phi
\end{align}

We assume $H_0$ and $\Delta \phi$ be gaussian distributions with means $\bar{H_0}, \bar{\Delta \Phi}$ and standard deviations $\sigma_{H}$, and $\sigma_{\Delta \phi}$. Then, we get

\begin{align}
P(\Delta t) &= \frac{1}{\sqrt{2\sigma_{H}^2\pi}}\frac{1}{\sqrt{2\sigma_{\Delta \Phi}^2\pi}}\int\int \delta(\Delta t - \frac{Q \Delta \phi}{H_0}) e^{-(\frac{H_0-\bar{H_0}}{\sigma_{H}})^2} e^{-(\frac{\Delta \Phi-\bar{\Delta \Phi}}{\sigma_{\Delta \Phi}})^2} dH_0 d\Delta \phi \\
&= \frac{1}{2\sigma_{H}\sigma_{\Delta \Phi}\pi Q} exp(\frac{-\bar{H_0}^2}{\sigma_H^2}+\frac{-(\bar{\Delta \Phi})^2}{\sigma_{\Delta \Phi}^2})
exp(\frac{(\frac{\bar{H_0}}{\sigma_H^2}+\frac{\bar{\Delta \phi}\Delta t}{Q\sigma_{\Delta \phi}^2})^2}{\frac{1}{\sigma_H^2}+\frac{\Delta t^2}{\sigma_{\Delta \phi}^2 Q^2}}) \frac{\frac{\bar{H_0}}{\sigma_H^2}+\frac{\bar{\Delta \phi}\Delta t}{Q\sigma_{\Delta \phi}^2}}{\frac{1}{\sigma_H^2}+\frac{\Delta t^2}{\sigma_{\Delta\phi}^2 Q^2}}
\sqrt{\frac{\pi}{\frac{1}{\sigma_H^2}+\frac{\Delta t^2}{\sigma_\phi^2Q^2}}}
\end{align}

If we further consider the information that the universe is expanding, which means $H_0$ is positive, the above formula will become

\begin{align}
\label{eqn:prior}
P(\Delta t) = \frac{1}{2\sigma_{H}\sigma_{\Delta \Phi}\pi Q} exp(\frac{-\bar{H_0}^2}{\sigma_H^2}+\frac{-(\bar{\Delta \Phi})^2}{\sigma_{\Delta \Phi}^2})
exp(\frac{(\frac{\bar{H_0}}{\sigma_H^2}+\frac{\bar{\Delta \phi}\Delta t}{Q\sigma_{\Delta \phi}^2})^2}{\frac{1}{\sigma_H^2}+\frac{\Delta t^2}{\sigma_{\Delta \phi}^2 Q^2}}) \frac{exp(-ab^2)+\sqrt{\pi a}b(erf(\sqrt{a}b)+1)}{2a}
\end{align}

 where $a=\frac{1}{\sigma_H^2}+\frac{\Delta t^2}{\sigma_\phi^2Q^2}$, $b=\frac{\frac{\bar{H_0}}{\sigma_H^2}+\frac{\bar{\Delta \phi}\Delta t}{Q\sigma_{\Delta \phi}^2}}{\frac{1}{\sigma_H^2}+\frac{\Delta t^2}{\sigma_{\Delta\phi}^2 Q^2}}$, and erf is defined as $\frac{2}{\sqrt(\pi)}\int_0^z exp(-t^2) dt$

For this paper, we assume $H_0$ is $70 \pm 7$, and we compute the prior $P(\Delta t)$ by formula \ref{eqn:prior}.

\subsection{Posterior}
The posterior can be computed by
$P(\Delta t \mid data) = P(data \mid \Delta t) P(\Delta t)$
% ----------------------------------------------------------------------

\section{Results}
\label{sec:results}
In this section, we'll show the analysis on gateway one data as an example, and we'll give a summary posterior plot for all ten gateway data. The analysis on other 9 gateway data  will be left to the appendix.

\subsection{Analysis on gateway one data}

We follow the methods described in section \ref{sec:method}.  For simplicity, in this subsection, I'll only show data that is whitened by whitening method 1.
\begin{enumerate}
\item Determine $\sigma_{intrinsic}$:
Fig \ref{fig:sigma1} shows how likelihood change with different $\sigma_{intrinsic}$ at different time delay. First, we find that there is a clear peak at $\sigma_{intrinsic}=0.202$, and this peak does not vary at different time delays. In general, we need to compute likelihood on $\sigma_{intrinsic}$ to time delay parameter spaces. However, since the optimal $\sigma_{intrinsic}$ seems not varying for different time delay, we can find the optimal $\sigma_{intrinsic}$ at a given time delay first and then sample on time delay parameter space.

\begin{figure}[!h]
\includegraphics[width=\textwidth, height=15cm, keepaspectratio]{sigma_0.png}
\caption{Likelihood to $\sigma_{Intrinsic}$ plot}
\label{fig:sigma1}
\end{figure}

\item Compute Prior Likelihood and Posterior:

After finding the optimal $\sigma_{intrinsic}$, we rescale the noise of light curves by $\sigma_{new}=\sqrt{\sigma_{intrinsic}^2+\sigma_{old}^2}$. Then, we compute the prior by formula \ref{eqn:prior} and the likelihood with the aid of PyCS code. The prior and the likelihood will be combined to get posterior. Fig \ref{fig:log_data1} shows the result.  There are couples of things to be noticed. First, the constraint on time delay is improved from prior to posterior. Second, though prior information gives us a loose constraint on time delay, it damps some of the fake peaks on likelihood and helps us identify the true time delay on posterior. For example, though there is a clear peak at true time  delay on likelihood, there are some fake outliers at  $\Delta t =140$ to $150$ and $\Delta t =-130$ to $-150$. The prior, though gives a loose constraint on time delay, damps these outliers, and helps us identified true time delay from posterior.

\begin{figure}[!h]
\includegraphics[width=\textwidth, height=15cm, keepaspectratio]{data1_full_log.png}
\caption{The top panel shows the likelihood distribution. The middle plot is the prior derived from equation \ref{eqn:prior}, and the bottom plot is the posterior $P(\Delta t \mid data)$ }. Dark dash line marks the correct time delay.
\label{fig:log_data1}
\end{figure}
\end{enumerate}


\subsection{Summary of all ten gateway data sets}
Fig \ref{fig:summary_post} and fig \ref{fig:summary_prior} show the posterior and prior distributions for all 10 data, which is whitened by whitening method 1. We also do the same analysis on data that is whitened by whitening method 2. The result is shown in Fig. \ref{fig:summary_post_newWhiten}.  Table  \ref{tab:summary}  summarize the results  for both two whitening method and the prior .

\begin{figure}[!h]
\includegraphics[width=\textwidth, height=15cm, keepaspectratio]{summary_posterior_summary.png}
\caption{Summary of posterior plot. We plot the posterior P($\Delta t \mid data$) for all ten gateway data cleaned by Whitening method 1.}
\label{fig:summary_post}
\end{figure}

\begin{figure}[!h]
\includegraphics[width=\textwidth, height=15cm, keepaspectratio]{summary_prior_summary.png}
\caption{Summary of prior plot. We plot the prior P($\Delta t$) for all ten gateway data.}
\label{fig:summary_prior}
\end{figure}

\begin{figure}[!h]
\includegraphics[width=\textwidth, height=15cm, keepaspectratio]{summary_posterior_summary_newWhiten.png}
\caption{Summary of posterior plot. We plot the posterior P($\Delta t \mid data$) for all ten gateway data cleaned by Whitening method 2.}
\label{fig:summary_post_newWhiten}
\end{figure}

The difference between two kinds of whitening method could be used to identify outliers. For example, we'll not trust the result if $\sigma\Delta t$($68\%$), the halfwidth of the $68\%$ credible region differs a lot. Also, if $\sigma\Delta t$($68\%$) is exceptionally small, say 1 day, the result might be dominate by the spike fluctuation in posterior. Besides, we don't expect to recover time delay larger than $120$, because the spline function will be able to fit light curves well if  they don't match with each other.  Based on these consideration, we define the following criteria for a good data:
\begin{enumerate}
\item $\sigma\Delta t$($68\%$) $>$ 1 days
\item The difference of $\sigma\Delta t$($68\%$) for whitening method 1 and whitening method 2 is smaller than 5 days
\item The measured time delay is smaller than 120 days. 
\end{enumerate}

We can also compute the precision, defined as 
\begin{equation}
P=\frac{1}{N} \Sigma_i (\frac{\lvert \delta t_{i, True} \rvert}{\Delta t_i})
\end{equation}
and the accuracy
\begin{equation}
A = \frac{1}{N} \Sigma_i (\frac{\lvert \Delta t_i-\Delta t_{i, True} \rvert}{\Delta t_i}),
\end{equation}
where N is number of data, $\Delta t_i$ is the median of posterior, and $\delta t_i$ is the error of time delay, estimated by halfwidth of $68\%$ confidence region of posterior.  

After rejecting the outliers, we get $A=0.022$ and $P=0.054$ for whitening method 1, and for whitening method 2 data, $A=0.024$ and $P=0.039$. 


\begin{table}[ht!]
\centering
\caption{The summary for ten gateway data analysis. $\delta \Delta t$ is the median of the difference between the measured time delay and the correct time delay. $\sigma\Delta t$($68\%$) is the halfwidth of the $68\%$ credible region.}
\label{tab:summary}
\begin{tabular}{|l|l|r|l|l|l|l|}
\hline
\multicolumn{1}{|c|}{\multirow{2}{*}{data}} & \multicolumn{2}{l|}{Prior}                                                    & \multicolumn{2}{l|}{Posterior (Whitening 1)}              & \multicolumn{2}{l|}{Posterior (Whitening 2)}              \\ \cline{2-7} 
\multicolumn{1}{|c|}{}                      & $\delta\Delta t$(days) & \multicolumn{1}{l|}{$\sigma\Delta t$(68$\%$) (days)} & $\delta\Delta t$ (days) & $\sigma\Delta t$(68$\%$) (days) & $\delta\Delta t$ (days) & $\sigma\Delta t$(68$\%$) (days) \\ \hline
1                                           & 2.97                   & 3.90                                                 & -0.34                   & 0.30                            & -1.54                   & 1.50                            \\ \hline
2                                           & -0.09                  & 2.85                                                 & -5.79                   & 7.36                            & -6.40                   & 0.30                            \\ \hline
3                                           & 0.60                   & 3.45                                                 & 1.50                    & 1.80                            & 0.90                    & 0.75                            \\ \hline
4                                           & 0.49                   & 1.65                                                 & -0.71                   & 1.35                            & -1.31                   & 0.90                            \\ \hline
5                                           & 6.65                   & 6.91                                                 & -2.96                   & 0.60                            & -4.46                   & 2.25                            \\ \hline
6                                           & -0.80                  & 4.05                                                 & -14.9                   & 1.35                            & -11.3                  & 10.8                            \\ \hline
7                                           & 1.92                   & 3.75                                                 & -0.18                   & 3.45                            & 0.72                    & 4.5                             \\ \hline
8                                           & 4.13                   & 7.96                                                 & 1.73                    & 0.15                            & 0.83                    & 2.10                            \\ \hline
9                                           & -3.18                  & 9.01                                                 & -23.9                   & 1.05                            & -18.49                  & 2.70                            \\ \hline
10                                          & -0.13                  & 3.00                                                 & -0.73                   & 0.15                            & -1.03                   & 0.15                            \\ \hline
\end{tabular}
\end{table}

\section{Discussion}
\label{sec:discussion}

First, we find that the result from whitening method 1 and whitening method 2 is comparable. It is because though whitening method 2 helps spline function to fit the data better, it also introduces additional fluctuation in the data and washes out some of the signal. Second, we find that even by this crude analysis, we can get time delay accuracy to $2\%$ and precision to $4\%$. We expect with the more subtle analysis, the good team could achieve much better accuracy and precision.  
% ----------------------------------------------------------------------

\section{Conclusions}
\label{sec:conclusions}
From the analysis, we draw the following conclusions.
\begin{itemize}
  \item The lens model prior do help us to identify the true time delay. 
  \item Both of the two whitening method we proposed are comparable. By combination of this two whitening method, we could identify some of the outliers. 
  \item We get $A=0.022$ and $P=0.054$ for whitening method 1, and $A=0.024$ and $P=0.039$ for whitening method 2. Though it still not achieves the goal, we expect the good team could do better than our simple model.  
\end{itemize}




% Include both collaboration papers and external citations:
\bibliography{lsstdesc,main}

\section{Appendix}
We'll show our analysis on all ten gateway data. 
\subsection{Gateway1 data}

\begin{figure}[!h]
  \centering
  \begin{minipage}[bottom]{0.4\textwidth}
\includegraphics[width=\textwidth, height=15cm, keepaspectratio]{whiten1/data1_full_log.png}
  \end{minipage}
  \hfill
  \begin{minipage}[bottom]{0.4\textwidth}
\includegraphics[width=\textwidth, height=15cm, keepaspectratio]{whiten2/data1_full_log.png}
  \end{minipage}
 \caption{Likelihood prior and posterior plot. The left panel is for data whitened by whitening method 1 and the right panel is whitened by whitening method 2. For each panel, the top panel shows the likelihood distribution. The middle plot is the prior derived from equation \ref{eqn:prior}, and the bottom plot is the posterior $P(\Delta t \mid data)$ . Dark dash line marks the correct time delay. }
\end{figure}
\newpage
\subsection{Gateway2 data}

\begin{figure}[!h]
  \centering
  \begin{minipage}[bottom]{0.4\textwidth}
\includegraphics[width=\textwidth, height=15cm, keepaspectratio]{whiten1/data2_full_log.png}
  \end{minipage}
  \hfill
  \begin{minipage}[bottom]{0.4\textwidth}
\includegraphics[width=\textwidth, height=15cm, keepaspectratio]{whiten2/data2_full_log.png}
  \end{minipage}
 \caption{Likelihood prior and posterior plot. The left panel is for data whitened by whitening method 1 and the right panel is whitened by whitening method 2. For each panel, the top panel shows the likelihood distribution. The middle plot is the prior derived from equation \ref{eqn:prior}, and the bottom plot is the posterior $P(\Delta t \mid data)$ . Dark dash line marks the correct time delay. }
\end{figure}
\newpage

\subsection{Gateway3 data}

\begin{figure}[!h]
  \centering
  \begin{minipage}[bottom]{0.4\textwidth}
\includegraphics[width=\textwidth, height=15cm, keepaspectratio]{whiten1/data3_full_log.png}
  \end{minipage}
  \hfill
  \begin{minipage}[bottom]{0.4\textwidth}
\includegraphics[width=\textwidth, height=15cm, keepaspectratio]{whiten2/data3_full_log.png}
  \end{minipage}
 \caption{Likelihood prior and posterior plot. The left panel is for data whitened by whitening method 1 and the right panel is whitened by whitening method 2. For each panel, the top panel shows the likelihood distribution. The middle plot is the prior derived from equation \ref{eqn:prior}, and the bottom plot is the posterior $P(\Delta t \mid data)$ . Dark dash line marks the correct time delay. }
\end{figure}
\newpage

\subsection{Gateway4 data}

\begin{figure}[!h]
  \centering
  \begin{minipage}[bottom]{0.4\textwidth}
\includegraphics[width=\textwidth, height=15cm, keepaspectratio]{whiten1/data4_full_log.png}
  \end{minipage}
  \hfill
  \begin{minipage}[bottom]{0.4\textwidth}
\includegraphics[width=\textwidth, height=15cm, keepaspectratio]{whiten2/data4_full_log.png}
  \end{minipage}
 \caption{Likelihood prior and posterior plot. The left panel is for data whitened by whitening method 1 and the right panel is whitened by whitening method 2. For each panel, the top panel shows the likelihood distribution. The middle plot is the prior derived from equation \ref{eqn:prior}, and the bottom plot is the posterior $P(\Delta t \mid data)$ . Dark dash line marks the correct time delay. }
\end{figure}
\newpage

\subsection{Gateway5 data}

\begin{figure}[!h]
  \centering
  \begin{minipage}[bottom]{0.4\textwidth}
\includegraphics[width=\textwidth, height=15cm, keepaspectratio]{whiten1/data5_full_log.png}
  \end{minipage}
  \hfill
  \begin{minipage}[bottom]{0.4\textwidth}
\includegraphics[width=\textwidth, height=15cm, keepaspectratio]{whiten2/data5_full_log.png}
  \end{minipage}
 \caption{Likelihood prior and posterior plot. The left panel is for data whitened by whitening method 1 and the right panel is whitened by whitening method 2. For each panel, the top panel shows the likelihood distribution. The middle plot is the prior derived from equation \ref{eqn:prior}, and the bottom plot is the posterior $P(\Delta t \mid data)$ . Dark dash line marks the correct time delay. }
\end{figure}
\newpage

\subsection{Gateway6 data}

\begin{figure}[!h]
  \centering
  \begin{minipage}[bottom]{0.4\textwidth}
\includegraphics[width=\textwidth, height=15cm, keepaspectratio]{whiten1/data6_full_log.png}
  \end{minipage}
  \hfill
  \begin{minipage}[bottom]{0.4\textwidth}
\includegraphics[width=\textwidth, height=15cm, keepaspectratio]{whiten2/data6_full_log.png}
  \end{minipage}
 \caption{Likelihood prior and posterior plot. The left panel is for data whitened by whitening method 1 and the right panel is whitened by whitening method 2. For each panel, the top panel shows the likelihood distribution. The middle plot is the prior derived from equation \ref{eqn:prior}, and the bottom plot is the posterior $P(\Delta t \mid data)$ . Dark dash line marks the correct time delay. }
\end{figure}
\newpage

\subsection{Gateway7 data}

\begin{figure}[!h]
  \centering
  \begin{minipage}[bottom]{0.4\textwidth}
\includegraphics[width=\textwidth, height=15cm, keepaspectratio]{whiten1/data7_full_log.png}
  \end{minipage}
  \hfill
  \begin{minipage}[bottom]{0.4\textwidth}
\includegraphics[width=\textwidth, height=15cm, keepaspectratio]{whiten2/data7_full_log.png}
  \end{minipage}
 \caption{Likelihood prior and posterior plot. The left panel is for data whitened by whitening method 1 and the right panel is whitened by whitening method 2. For each panel, the top panel shows the likelihood distribution. The middle plot is the prior derived from equation \ref{eqn:prior}, and the bottom plot is the posterior $P(\Delta t \mid data)$ . Dark dash line marks the correct time delay. }
\end{figure}
\newpage

\subsection{Gateway8 data}

\begin{figure}[!h]
  \centering
  \begin{minipage}[bottom]{0.4\textwidth}
\includegraphics[width=\textwidth, height=15cm, keepaspectratio]{whiten1/data8_full_log.png}
  \end{minipage}
  \hfill
  \begin{minipage}[bottom]{0.4\textwidth}
\includegraphics[width=\textwidth, height=15cm, keepaspectratio]{whiten2/data8_full_log.png}
  \end{minipage}
 \caption{Likelihood prior and posterior plot. The left panel is for data whitened by whitening method 1 and the right panel is whitened by whitening method 2. For each panel, the top panel shows the likelihood distribution. The middle plot is the prior derived from equation \ref{eqn:prior}, and the bottom plot is the posterior $P(\Delta t \mid data)$ . Dark dash line marks the correct time delay. }
\end{figure}
\newpage

\subsection{Gateway9 data}

\begin{figure}[!h]
  \centering
  \begin{minipage}[bottom]{0.4\textwidth}
\includegraphics[width=\textwidth, height=15cm, keepaspectratio]{whiten1/data9_full_log.png}
  \end{minipage}
  \hfill
  \begin{minipage}[bottom]{0.4\textwidth}
\includegraphics[width=\textwidth, height=15cm, keepaspectratio]{whiten2/data9_full_log.png}
  \end{minipage}
 \caption{Likelihood prior and posterior plot. The left panel is for data whitened by whitening method 1 and the right panel is whitened by whitening method 2. For each panel, the top panel shows the likelihood distribution. The middle plot is the prior derived from equation \ref{eqn:prior}, and the bottom plot is the posterior $P(\Delta t \mid data)$ . Dark dash line marks the correct time delay. }
\end{figure}
\newpage

\subsection{Gateway10 data}

\begin{figure}[!h]
  \centering
  \begin{minipage}[bottom]{0.4\textwidth}
\includegraphics[width=\textwidth, height=15cm, keepaspectratio]{whiten1/data10_full_log.png}
  \end{minipage}
  \hfill
  \begin{minipage}[bottom]{0.4\textwidth}
\includegraphics[width=\textwidth, height=15cm, keepaspectratio]{whiten2/data10_full_log.png}
  \end{minipage}
 \caption{Likelihood prior and posterior plot. The left panel is for data whitened by whitening method 1 and the right panel is whitened by whitening method 2. For each panel, the top panel shows the likelihood distribution. The middle plot is the prior derived from equation \ref{eqn:prior}, and the bottom plot is the posterior $P(\Delta t \mid data)$ . Dark dash line marks the correct time delay. }
\end{figure}


\end{document}
% ======================================================================
%
